%%%%%%%%%%%%%%%%%%%%%%%%%%%%%%%%%%%%%%%%%
% Plasmati Graduate CV
% LaTeX Template
% Version 1.0 (24/3/13)
%
% This template has been downloaded from:
% http://www.LaTeXTemplates.com
%
% Original author:
% Alessandro Plasmati (alessandro.plasmati@gmail.com)
%
% License:
% CC BY-NC-SA 3.0 (http://creativecommons.org/licenses/by-nc-sa/3.0/)
%
% Important note:
% This template needs to be compiled with XeLaTeX.
% The main document font is called Fontin and can be downloaded for free
% from here: http://www.exljbris.com/fontin.html
%
%%%%%%%%%%%%%%%%%%%%%%%%%%%%%%%%%%%%%%%%%

%----------------------------------------------------------------------------------------
%	PACKAGES AND OTHER DOCUMENT CONFIGURATIONS
%----------------------------------------------------------------------------------------

\documentclass[a4paper,10pt]{extarticle} % Default font size and paper size

\usepackage{fontspec} % For loading fonts
\defaultfontfeatures{Mapping=tex-text}
% \setmainfont[SmallCapsFont = Fontin SmallCaps]{Fontin} % Main document font
% \fontspec{[FontAwesome.otf]}
\setmainfont[Path = ./Ubuntu/,  %% Optional; but UPDATE this if 
                         %% your font files are in a folder
 Extension = .ttf,
 UprightFont = *-Regular,
 BoldFont = *-Bold,
 ItalicFont = *-Italic,
 SmallCapsFont = *-Medium]
{Ubuntu}
\fontspec{[fontawesome-webfont.ttf]}

\usepackage{color}
\definecolor{primary}{RGB}{107, 16, 86}
\definecolor{secondary}{RGB}{0, 0, 0}

\usepackage{xunicode,xltxtra,url,parskip} % Formatting packages

\usepackage[usenames,dvipsnames]{xcolor} % Required for specifying custom colors

%\usepackage[big]{layaureo} % Margin formatting of the A4 page, an alternative to layaureo can be 
%\usepackage{fullpage}
\usepackage{geometry}
\geometry{a4paper,margin=0.40cm}
%\geometry{a4paper,left=20mm, top=20mm}
 %To reduce the height of the top margin uncomment: \addtolength{\voffset}{-1.3cm}

\usepackage{hyperref} % Required for adding links	and customizing them
\definecolor{linkcolour}{rgb}{0,0.4,0.5} % Link color
% \definecolor{linkcolour}{rgb}{0.3,0.3,0.3} % Link color
\hypersetup{colorlinks,breaklinks,urlcolor=linkcolour,linkcolor=linkcolour} % Set link colors throughout the document

\usepackage{titlesec} % Used to customize the \section command
\titleformat{\section}{\large\scshape\raggedright}{}{0em}{}[\titlerule] % Text formatting of sections
\titlespacing{\section}{0pt}{0pt}{0pt} % Spacing around sections

\usepackage{multicol}
\setlength{\columnsep}{0cm}

\usepackage{tabularx}

\usepackage{textcomp}

\usepackage{fontawesome}

\usepackage{enumitem}
\setlist[description]{%
  topsep=10pt,               % space before start / after end of list
  itemsep=1pt,               % space between items
%  font={\bfseries\sffamily\color{red}}, % if colour is needed
}

\def\arraystretch{1}
\renewcommand{\baselinestretch}{1.1}

\begin{document}

\pagestyle{empty} % Removes page numbering

%\font\fb=''[cmr10]'' % Change the font of the \LaTeX command under the skills section

%----------------------------------------------------------------------------------------
%	NAME AND CONTACT INFORMATION
%----------------------------------------------------------------------------------------
\begin{multicols}{3}
% \par{\centering\normalsize {\textsc{Undergraduate Student At Indian Institute of Technology, Kharagpur}}\par}\normalsize
% \par{\centering\normalsize {\textsc{Department of Computer Science and Engineering}}\par}\normalsize
%\par{{\begin{center}Dual Degree, \emph{Computer Science and Engineering}\end{center}}}
\normalsize  \faGlobe\ {\href{https://thealphadollar.github.io/}{\  thealphadollar.github.io}}\\
\normalsize \faGithub\ {\href{https://github.com/thealphadollar}{\  thealphadollar}}\\
\normalsize  \faLinkedinSquare\ {\href{https://www.linkedin.com/in/thealphadollar}{\  thealphadollar}}\\
\columnbreak
\normalsize\par{\centering{\huge\textsc{\textcolor{primary}{Shivam Kumar Jha}}}\par} % Your name
\par{\centering\normalsize {\textsc{BRH Hall of Residence, IIT Kharagpur, WB, India - 721302}}\hfill\par}
\columnbreak
\raggedright\hfill\normalsize \faEnvelope\ {\href{mailto:shivam.cs.iit.kgp@gmail.com}{\  shivam.cs.iit.kgp@gmail.com}}\\
\raggedright\hfill{\faPhone\ \  +91-7830380698}
\end{multicols}

%----------------------------------------------------------------------------------------
%	EDUCATION
%----------------------------------------------------------------------------------------

\vspace{-0.6cm}
\section{\textcolor{primary}{Academics}}

\begin{tabular}{r|p{17.5cm}}	
\textbf{2017-2022} & \textit{B.Tech + M.Tech (Dual Degree)} in \textbf{Computer Science and Engineering, IIT Kharagpur}\\
\hfill GPA & \textbf{8.54}/10.0 (Ongoing)\\
% \textbf{2017} & \textit{Higher Secondary School Certificate Examination}, \textbf{CBSE}\\
% \hfill GPA & \textbf{91\%}\\
% \textbf{2015} & \textit{Secondary School Certificate Examination}, \textbf{CBSE}\\
% \hfill GPA & \textbf{10}/10 \\
\end{tabular}

%----------------------------------------------------------------------------------------
%	SKILLS 
%----------------------------------------------------------------------------------------

\section{\textcolor{primary}{Technical Skills}}

\begin{tabular}{r|p{15cm}}
\textsc{Programming / Scripting Languages} & C | C++ | C\# | JavaScript | Golang | Python | Java | PHP \\
\textsc{Libraries / Frameworks} & Node.js | OpenCV | Click | .NET Core | Xamarin \\
\textsc{DBMS} & SQLite | MongoDB | PostgreSQL | MySQL \\
\textsc{Systems / Platforms} & Linux | Docker | Vagrant | Heroku | Azure | GCP \\
\textsc{Web Technologies} & Jekyll | Foundation 6 | Bootstrap | Flask | Django | ASP.NET | ReactJS | AngularJS | REST | GraphQL | Nginx \\
% \textsc{Others} & Unity Game Engine (Boo Script) | Latex | SolidWorks | MATLAB
\end{tabular}

%----------------------------------------------------------------------------------------
%	EXPERIENCE
%----------------------------------------------------------------------------------------

\section{\textcolor{primary}{Experience}}

\begin{tabularx}{\linewidth}{ l | X }

\textsc{May 19} & \textbf{SE Intern: {\href{https://cleartax.in/}{\ Cleartax}}}\hfill\textbf{ClearTax, Bangalore}\\
\textsc{July 19}& {- Updated information in micro-services (namely IFSCParser) for 2019 July filing season.}\\
& {- Added and updated fields for ITR-2 2019 filing in data-layer, controllers and views, and established backward compatibility with previous filings.}\\
& {- Worked on making tax-filing smoother by adding numerous bug-fixes, validation optimization's and front-end improvements.}\\

\textsc{Nov 18} & \textbf{SE Intern: {\href{https://vwo.com/}{\ VWO}}}\hfill\textbf{Wingify, New Delhi}\\
\textsc{Jan 18}& {- Dockerised core components of VWO to facilitate easy migration and production setup.}\\
& {- Resolved data handling errors and UI fixes in VWO Dashboard, written using AngularJS.}\\
& {- Improved data synchronization speed and efficacy between front-end and data layer.}\\

\textsc{May 18} & \textbf{Student Developer: {\href{https://github.com/thealphadollar/Nephos}{\ Project Nephos}}}\hfill\textbf{CCExtractor, Google Summer Of Code 2018}\\
\textsc{Aug 18}& {- A command line tool to automate the process of recording, processing and uploading IPTV streams for Universities under \href{http://www.redhenlab.org/}{RedHen Labs}, project closely supported by University Of Nevarra, Spain.}\\
& {- Learned extensive error handling, database management, remote server operations, FTP handling in Python, scheduling, modular configuration implementation, extensive logging and \href{https://thealphadollar.github.io/learning/2019/05/13/go-dry.html}{DRY methods}.}\\

%\multicolumn{2}{c}{} \\
\end{tabularx}

%----------------------------------------------------------------------------------------
%	Projects
%----------------------------------------------------------------------------------------
\vspace{-0.1cm}
\section{\textcolor{primary}{\href{https://www.github.com/thealphadollar}{Projects}}}
\vspace{-0.6cm}
\begin{tabular}{p{19.7cm}}
% \fontsize{9}{12}\selectfont{
\begin{description}[style=nextline, font=$\bullet$\hspace{2mm}\normalsize]

 \item[{\href{https://github.com/CCExtractor/sample-platform}{Sample-Platform}, Google Summer of Code 2019}] Continuous Integration platform for \href{https://github.com/CCExtractor/ccextractor}{CCExtractor's principal repository}: improved the code sanity checking mechanisms through mypy, dodgy and other tools, increased unit-test coverage from 60.2\% to 90+\%, setup database migration methods, improved file comparison algorithms, implemented parallel running of tasks, {\href{https://github.com/CCExtractor/sample-platform/pulls?utf8=\%E2\%9C\%93&q=is\%3Apr+author\%3Athealphadollar+}{and more}}.
 
 \item[{\href{https://github.com/thealphadollar/messiah}{Messiah}, Microsoft CodeFunDo++ 2018}] 
 A service that predicts the magnitude of Earthquake using Tensorflow. Had the opportunity to learn optimization of web services; backend and frontend, reducing loading time by 2.5x. Later developed in {\href{https://www.linkedin.com/company/streethack/?originalSubdomain=in}{Streethack Blockhack}} under open category and bagged Runners Up position.
 
 \item[{\href{https://github.com/thealphadollar/opensoft18}{DigiCon}, OpenSoft
 2018 IIT Kharagpur}] Gold winning entry (out of 13 entries) to the Inter Hall
 Opensoft, IIT Kharagpur. The project intelligently parses different
 parts of a doctor's prescription (written by hand) and is an amalgamation of multiple technologies
 such as OpenCV, Flask, Natural Language Processing, REST API, bash scripting and Docker.
 
%  \item[Games Using PyGame And Python] Worked on two projects; \href{https://github.com/thealphadollar/AirHockey}{\ AirHockey} and \href{https://github.com/thealphadollar/brkout}{BrkOut}: implementing basic mechanisms of physics programming and graphic development along with elements to cryptography.
 
%  \item[\href{https://github.com/thealphadollar/salvator}{Salvator}: Automated Task of Birthday Wishing]Salvator is a task-automation bot; sends birthday wishes and implements an aesthetic CLI.

%  \item[\href{https://github.com/thealphadollar/send_event_invites}{Send Event Invites}: Sending Invites Using CLI]A CLI tool for sending event invitations to a large number of attendees. The tool has been made in such a manner that it is very easy to add new calendar APIs and use them.

%  \item[Github OpenSource Projects] Worked on a lot of opensource projects: \href{https://github.com/thealphadollar/AirHockey}{\ AirHockey}, \href{https://github.com/thealphadollar/brkout}{BrkOut}, \href{https://github.com/thealphadollar/salvator}{Salvator}, \href{https://github.com/thealphadollar/set_proxy}{Set Proxy} and \href{https://github.com/thealphadollar?tab=repositories}{many more}.
\end{description}
% }
\end{tabular}

%----------------------------------------------------------------------------------------
%	Activities & Leadership
%----------------------------------------------------------------------------------------
\vspace{-0.6cm}
\section{\textcolor{primary}{Activities \& Leadership}}

\begin{itemize}[leftmargin=0.55cm, rightmargin=0.2cm, label={\Large\textbullet}]

\item \textbf{Student's Technology Coordinator, Technology GymKhana, IIT Kharagpur}: Holding the highest nominated technical position in the institute, I'm responsible for the smooth building and development of all technological fronts; \href{http://gymkhana.iitkgp.ac.in/index.php}{GymKhana website} maintenance, Institute's app development, students' grievance portal, etc.

\item \textbf{OpenSource Program Involvement}: Mentored high school students for Google CodeIn 2018 under \href{https://codein.withgoogle.com/organizations/ccextractor-development/}{CCExtractor} and was a top mentor for GirlScript Summer of Code 2018 and \href{https://oss2019.github.io/SoI.html}{Summer of Innovation, IIT Dharwad}. Successfully conducted \href{https://gssoc.tech/}{GirlScript Summer of Code 2019} and \href{https://kwoc.kossiitkgp.org/}{Kharagpur Winter of Code 2018}; responsible for the development, deployment and maintenance of the website (full-stack).
% \item \textbf{Founder, \href{https://www.facebook.com/codestashkgp/}{CodeStash IIT Kharagpur}}: Created a community in my freshman year to promote competitive coding and increase awareness about various fields of computer science (inspiration from code.org).

\item \textbf{Student Speaker, \href{http://pragmaconf.tech}{Pragma IIIT Allahabad}}: Spoke about \href{https://prezi.com/view/tf50MBbGtm9FgPKQfieI/}{efficient methods of planning and programming for perpetual processes} in order to avoid overhead costs of maintenance and denial of service. 

\item \textbf{Maintainer, \href{https://wiki.metakgp.org/w/Metakgp:About}{MetaKGP IIT Kharagpur} | Executive Member, \href{https://kossiitkgp.org/about/index.html}{KOSS IIT Kharagpur}}: Responsible for fostering participation, promoting the growth of open-source culture and maintaining open-source (campus-centric) APIs, services and task-bots.
% \item \textbf{Secretary, CodeClub IIT Kharagpur}: Developed and managed website for Bitwise 2018 and \href{http://upai-summit.in/}{up.AI 2018}, West Bengal's largest Artificial Intelligence and Machine-Learning summit.
\end{itemize}


%----------------------------------------------------------------------------------------
%	Technical Interests
%----------------------------------------------------------------------------------------

% \vspace{-0.4cm}
% \section{\textcolor{primary}{Technical Interests}}

% \noindent Cloud Computing, Bot Creation And Task Automation, Cryptography, Game Development, Machine Learning, Big Data, Linux \\

%----------------------------------------------------------------------------------------
%	Scholastic Achievements
%----------------------------------------------------------------------------------------

% \vspace{-0.1cm}
% \section{\textcolor{primary}{Scholastic Achievements}}

% %\begin{multicols}{2}
% - \textbf{IITJEE Advanced \emph{AIR 286}:} Under top 0.15\% amongst more than 2,00,000 students \\
% - \textbf{IITJEE Mains \emph{AIR 2285}:} In top 0.2\% amongst more than 11,00,000 students. \\
% - \textbf{National Olympiads:} KVPY SA AIR , KVPY SX Scholar, Indian National Physics Olympiad Stage II, Indian National Astronomy Olympiad Stage II, Indian National Chemistry Olympiad Stage II. \\
% - \textbf{RMSEE \emph{AIR 56}:} Ex-Georgian, Student of Rashtriya Military School Chail, Himanchal Pradesh, India. \\
%\end{multicols}

%----------------------------------------------------------------------------------------

%----------------------------------------------------------------------------------------
%	COURSEWORK
%----------------------------------------------------------------------------------------
% \vspace{-0.4cm}
% \section{\textcolor{primary}{Relevant Coursework}
% \hfill\small\textsc{(T)heory and (L)aboratory}}

% \begin{tabular}{r|p{15cm}}
% \textsc{Completed} & Programming And Data Structures (T/L) | Algorithms I (T/L) | Discrete Structures (T) | Formal Language Automata Theory (T) | Software Engineering (T/L) | Switching Circuits And Logic (T/L) | Probability And Statistics (T) \\
% \textsc{OnGoing} & Computer Organization And Architecture (T/L) | Algorithms II (T) | Compilers (T/L) | Linear Algebra (T)
% \end{tabular}

% ----------------------------------------------------------------------------------------

\end{document}

